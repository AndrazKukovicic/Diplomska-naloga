V diplomski nalogi bom predstavil osnovne pojme univerzalne algebre skupaj s primeri, ki smo jih v dosedanjem študiju
matematike že srečali. Postopno bom predstavil pojme in rezultate, ki bodo vodili do srža diplomske naloge:
formulaciji in dokazu Mal'cevega izreka in terma, ki se imenuje po njem.
Eden od ciljev univerzalne algebre je iskanje in uporaba skupnih lastnosti na prvi pogled popolnoma
razližnih algebrskih struktur. To nam da nov, splošnejši in poenoten pogled na že znane strukture, hkrati pa lahko 
posamezne primere na ta način predstavimo bolj ekonomično, z manj pisanja. 
Spomnimo se naprimer primera dokazovanja izreka o izomorfizmu, ki smo ga v pri Algebri II prvič dokazali za grupe, nato 
smo ga formulirali še za druge strukture npr. kolobarje, algebre, module. Pri tem smo običajno rekli, da dokaz poteka 
enako le da ustrezno spremenimo oznake. Z uporabo univerzalne algebre bomo lahko tak dokaz za več različnih struktur 
naredili popolnoma formalno. 
Pred definicijo in poglobitvijo v glavno temo naloge si bomo ogledali še nekaj osnovnih rezultatov o mrežah, ki bodo 
kasneje uporabljeni v nekaterih izrekih in dokazih. 

Mreže 

Definicija: Neprazna množica $M$ z dvema binarnima operacijama, ki ju označimo z $\vee$ in $\wedge$ je \emph{mreža}, če 
za vsaka dva elementa iz $M$ veljajo naslednjim pogojem:\\
\begin{enumerate} 
    \item[M1] $x \vee y = y \vee x$ in $x \wedge y = y \wedge x$ (komutativnost)
    \item[M2] $x\vee \left(y \vee z\right) = \left(x \vee y\right)\vee z$ in 
    $\wedge \left(y \wedge z\right) = \left(x \wedge y\right)\wedge z$ (asociativnost)
    \item[M3] $x \vee x = x$ in $x \wedge x = x$ (idempotentnost)
    \item[M4] $x = x \vee \left(x \wedge y\right)$ in $x = x \wedge \left(x \vee y\right)$ (absorbcijsko pravilo).\\
\end{enumerate}

Pokažimo sedaj, da je definicija mreže ekvivalentna naslednji trditvi.
Trditev: Delno urejena množica $M$ je mreža natanko takrat, ko za vsaka $x,y \in M$, v $M$ obstajata $\sup{x,y}$ in 
$inf{x,y}$.
Dokaz: Spomnimo se najprej kaj je delno urejena množica. To je množica $m$ z relacijo $\leq$, da za vse elemente 
$x,y, z \in M$ velja: \begin{enumerate}
    \item $x \leq x$ (refleksivnost)
    \item Če je $x \leq y$ in $y \leq x$, sledi $x = y$. (antisimetričnost)
    \item Če je $x \leq y$ in $y \leq z$, sledi $x \leq z$. (tranzitivnost)
\end{enumerate}\\
Dokazali bomo implikaciji v obe smeri. Najprej predpostavimo, da je $M$ mreža po definiciji in dokžimo trditev. 
Definirajmo relacijo $\leq$: za $a, b \in M$ velja $a \leq b$ natanko takrat, ko je $a = a \wedge b$. 
Najprej bomo preverili, da tako definirana relacija podaja delno ureditev na $M$, nato pa še obstoj $sup{a,b}$ in 
$inf{a,b}$.
Za prvi del moramo preveriti refleksivnost, antisimetričnost in tranzitivnost. 
\begin{itemize}
    \item[(r)] Ker je $a \wedge a = a$ sledi $a \leq a$.
    \item[(a)] Če je $a\leq b$ in $b \leq a$ to pomeni, da je $a = a \wedge b$ in $b = b \wedge a$, torej je $a = b$.
    \item[(t)] Če je  $a\leq b$ in $b \leq c$ to pomeni, da je $a = a \wedge b$ in $b = b \wedge c$, torej je 
    $a = a\wedge b = a\wedge \left(b \wedge c\right)$ zaradi asociativnosti je slednje enako $\left(a \wedge b\right) 
    \wedge c = a \wedge c$. Dobili smo $a = a \wedge c$, od tod pa po definiciji relacije $\leq$ $a \leq c$.
\end{itemize}
Pokazali smo, da je $M$ z relacijo $\leq$ delno urejena. Sedaj pokažimo še obsotoj supremov in infimov. 
Iz $a = a \wedge \left(a \vee b\right)$ in $b = b \wedge \left(a \vee b\right)$ sledi $ a \leq a \vee b$ in $ b \leq
a \vee b$ torej je $a \vee b$ zgornja meja tako za $a$, kot za $b$. Pokažimo še, da je $a \vee b$ najmanjša zgornja meja
Če je $a \leq u$ in $b \leq u$ z neki $u \in M$, je po definiciji $\leq$ $a = a \wedge u$ in zato $a \vee u = 
\left(a \wedge u \right) \vee u = u$ in podobno $b \vee u = u$, torej je $\left(a \wedge u\right) \wedge 
\left(b \wedge u\right) = u \wedge u = u$. In zato $\left(a \vee b \right) \vee u = u$, od tu pa vstavimo $u$ v
 $ \left(a \vee b\right) \wedge u = \left(a \vee b\right) \wedge \left( \left(a \vee b\right) \vee u\right) = 
 a \vee b$ pri zadnji enakosti smo uporabili absorbcijsko pravilo. To nam da $ a \vee b \leq u$. Torej je res 
 $a \vee b = sup{a, b}$. Podobno pokažemo, da je $a \wedge b = inf{a, b}$.\\
 Dokažimo še obratno implikacijo torej, da iz trditve sledi definicija. Predpostavimo, da drži naša trditev in 
 definirajmo operaciji $\wedge$ in $\vee$ takole: $a \wedge b := inf{a, b}$ in $a \vee b := sup{a, b}$.
 Pokažimo, da tako definirana operacija $\vee$ zadošča pogojem M1 - M4.
 \begin{enumerate}
    \item[M1] Res velja $sup{a, b} = sup{b, a}$.
    \item[M2] Imamo $a \vee \left(b \vee c\right)$ torej je $sup{a, sup{b, c}}$, kar je enako $sup{sup{a,b}, c}$ in zato 
    $a \vee \left(b \vee c\right) = \left(a \vee b\right) \vee c$. 
    \item[M3] Ker je $a = sup{a, a}$ je res $a \vee a = a$. 
    \item[M4] $a = a \vee \left(a \wedge b\right)$ zapišemo kot $a = sup {a, inf{a, b}}$ kar je res, saj je $inf{a, b} 
    \leq a$. 
 \end{enumerate} 
Podobno preverimo za operacijo $\wedge$.