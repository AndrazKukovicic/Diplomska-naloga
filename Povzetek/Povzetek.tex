\documentclass[a4paper,11pt]{article}

\usepackage[slovene]{babel}
\usepackage{amsfonts,amssymb,amsmath,amsthm}
\usepackage[utf8]{inputenc}
\usepackage[T1]{fontenc}
\usepackage{lmodern}
\usepackage{graphicx}
\usepackage{biblatex}
\addbibresource{literatura.bib}
\def\qed{$\hfill\Box$}   % konec dokaza
\def\qedm{\qquad\Box}   % konec dokaza v matematičnem načinu
\newtheorem{izrek}{Izrek}
\newtheorem{trditev}{Trditev}
\newtheorem{posledica}{Posledica}
\newtheorem{lema}{Lema}
\newtheorem{pripomba}{Pripomba}
\newtheorem{definicija}{Definicija}
\newtheorem{zgled}{Zgled}
\newtheorem{primer}{Primer}


\title{Uvod v univerzalno algebro in Mal'cev pogoj \\ 
\Large Povzetek diplomske naloge}
\author{Andraž Kukovičič \\
Mentorica: izr. prof. dr. Ganna Kudryavtseva}
\date{Študijsko leto 2024/25}

\begin{document}


%%%%%%%%%%%%%%%%%%%%%%%%%%%%%%%%%%%%%%%%%%%%%%%%%%%%%%%%%%%%%%%%%%%%%


\maketitle

\section{Uvod}
V diplomski nalogi in njenem povzetku bom predstavil osnovne pojme univerzalne algebre skupaj s primeri, 
ki smo jih v dosedanjem študiju matematike že srečali. 
Postopno bom predstavil pojme in rezultate, ki bodo vodili do srža diplomske naloge:
formulacije in dokaza Mal'cevega izreka in terma, ki se imenuje po njem.
Eden od ciljev univerzalne algebre je iskanje in uporaba skupnih lastnosti na prvi pogled popolnoma
različnih algebrskih struktur. To nam da nov, splošnejši in poenoten pogled na že znane strukture, hkrati pa lahko 
posamezne primere na ta način predstavimo bolj ekonomično, z manj pisanja. 
Spomnimo se naprimer dokazovanja izreka o izomorfizmu, ki smo ga pri Algebri II prvič dokazali za grupe, nato 
smo ga formulirali še za druge strukture npr. kolobarje, algebre, module. Pri tem smo običajno rekli, da dokaz poteka 
enako le da ustrezno spremenimo oznake. Z uporabo univerzalne algebre bomo lahko tak dokaz za več različnih struktur 
naredili popolnoma formalno. 
Pred definicijo in poglobitvijo v glavno temo naloge si bomo ogledali še nekaj osnovnih rezultatov o mrežah, ki bodo 
kasneje uporabljeni v nekaterih izrekih in dokazih.

\section{Mreže}
\begin{definicija}
    Neprazna množica $M$ z dvema binarnima operacijama, ki ju označimo z $\vee$ in $\wedge$ je \emph{mreža}, če 
za vsaka dva elementa iz $M$ veljajo naslednjim pogojem:\\
\begin{enumerate} 
    \item[$$M1$$] $x \vee y = y \vee x$ in $x \wedge y = y \wedge x$ (komutativnost)
    \item[$$M2$$] $x\vee \left(y \vee z\right) = \left(x \vee y\right)\vee z$ in 
    $x \wedge \left(y \wedge z\right) = \left(x \wedge y\right)\wedge z$ (asociativnost)
    \item[$$M3$$] $x \vee x = x$ in $x \wedge x = x$ (idempotentnost)
    \item[$$M4$$] $x = x \vee \left(x \wedge y\right)$ in $x = x \wedge \left(x \vee y\right)$ (absorbcijsko pravilo).\\
\end{enumerate}
\end{definicija}
Pokažimo sedaj, da je definicija mreže ekvivalentna naslednji trditvi.

\begin{trditev}
Delno urejena množica $M$ je mreža natanko takrat, ko za vsaka $x,y \in M$, v $M$ obstajata $\sup\left\{x,y\right\}$ in 
$inf\left\{x,y\right\}$.
\end{trditev}

\begin{proof}
Spomnimo se najprej kaj je delno urejena množica. To je množica $m$ z relacijo $\leq$, da za vse elemente 
$x,y, z \in M$ velja: \begin{enumerate}
    \item $x \leq x$ (refleksivnost)
    \item Če je $x \leq y$ in $y \leq x$, sledi $x = y$. (antisimetričnost)
    \item Če je $x \leq y$ in $y \leq z$, sledi $x \leq z$. (tranzitivnost)
\end{enumerate}\\
Dokazali bomo implikaciji v obe smeri. Najprej predpostavimo, da je $M$ mreža po definiciji in dokažimo trditev. 
Definirajmo relacijo $\leq$: za $a, b \in M$ velja $a \leq b$ natanko takrat, ko je $a = a \wedge b$. 
Najprej bomo preverili, da tako definirana relacija podaja delno ureditev na $M$, nato pa še obstoj $sup\left\{a,
b\right\}$ in $inf\left\{a,b\right\}$.
Za prvi del moramo preveriti refleksivnost, antisimetričnost in tranzitivnost. 
\begin{itemize}
    \item[(r)] Ker je $a \wedge a = a$ sledi $a \leq a$.
    \item[(a)] Če je $a\leq b$ in $b \leq a$ to pomeni, da je $a = a \wedge b$ in $b = b \wedge a$, torej je $a = b$.
    \item[(t)] Če je  $a\leq b$ in $b \leq c$ to pomeni, da je $a = a \wedge b$ in $b = b \wedge c$, torej je 
    $a = a\wedge b = a\wedge \left(b \wedge c\right)$ zaradi asociativnosti je slednje enako $\left(a \wedge b\right) 
    \wedge c = a \wedge c$. Dobili smo $a = a \wedge c$, od tod pa po definiciji relacije $a \leq c$.
\end{itemize}
Pokazali smo, da je $M$ z relacijo $\leq$ delno urejena. Sedaj pokažimo še obsotoj supremov in infimov. 
Iz $a = a \wedge \left(a \vee b\right)$ in $b = b \wedge \left(a \vee b\right)$ sledi $ a \leq a \vee b$ in $ b \leq
a \vee b$ torej je $a \vee b$ zgornja meja tako za $a$, kot za $b$. Pokažimo še, da je $a \vee b$ najmanjša zgornja meja
Če je $a \leq u$ in $b \leq u$ z neki $u \in M$, je po definiciji relacije $\leq$: $a = a \wedge u$ in zato $a \vee u = 
\left(a \wedge u \right) \vee u = u$ in podobno $b \vee u = u$, torej je $\left(a \wedge u\right) \wedge 
\left(b \wedge u\right) = u \wedge u = u$. In zato $\left(a \vee b \right) \vee u = u$, od tu pa vstavimo $u$ v
 $ \left(a \vee b\right) \wedge u = \left(a \vee b\right) \wedge \left( \left(a \vee b\right) \vee u\right) = 
 a \vee b$ pri zadnji enakosti smo uporabili absorbcijsko pravilo. To nam da $ a \vee b \leq u$. Torej je res 
 $a \vee b = sup\left\{a, b\right\}$. Podobno pokažemo, da je $a \wedge b = inf\left\{a, b\right\}$.\\
 Dokažimo še obratno implikacijo torej, da iz trditve sledi definicija. Predpostavimo, da drži naša trditev in 
 definirajmo operaciji $\wedge$ in $\vee$ takole: $a \wedge b := inf\left\{a, b\right\}$ in $a \vee b := sup\left\{a, b\right\}$.
 Pokažimo, da tako definirana operacija $\vee$ zadošča pogojem $M1$ - $M4$.
 \begin{enumerate}
    \item[$M1$] Res velja $sup\left\{a, b\right\} = sup\left\{b, a\right\}$.
    \item[$M2$] Imamo $a \vee \left(b \vee c\right)$ torej je $sup\left\{a, sup\left\{b, c\right\}\right\}$, kar je enako
    $sup\left\{sup\left\{a,b\right\}, c\right\}$ in zato 
    $a \vee \left(b \vee c\right) = \left(a \vee b\right) \vee c$. 
    \item[$M3$] Ker je $a = sup\left\{a, a\right\}$ je res $a \vee a = a$. 
    \item[$M4$] $a = a \vee \left(a \wedge b\right)$ zapišemo kot $a = sup \left\{a, inf\left\{a, b\right\}\right\}$ 
    kar je res, saj je $inf\left\{a, b\right\} \leq a$. 
 \end{enumerate} 
Podobno preverimo za operacijo $\wedge$.
\end{proof}

\section{Algebra}
Poglejmo si sedaj definicijo algebre. Najprej definirajmo tip ali jezik algebre. 

\begin{definicija} \emph{Tip} ali \emph{jezik} algebre $A$ je množica funkcij oziroma operacij $f: A^n \rightarrow A$, 
označimo jo z $\mathcal{F}$. Število $n \in \mathbb{N}_0$ imeneujemo \emph{rang} ali -arnost funkcije $f$.
\end{definicija}

Vsaka funkcija $f \in \mathcal{F}$ ima prirejeni rang. 

Sledi definicija algebre, ki bo zaobjela večino znanih algebrskih struktur, zato se na prvi pogled zdi zapletena, 
vendar bomo kasneje skozi primere videli, da lahko algebrske strukture enostavno predstavimo kot primere algeber. 

\begin{definicija} \emph{Algebra $A$} tipa $\mathcal{F}$ je urejeni par $\left(A, \mathcal{F}\right)$, kjer je $A$ neprazna 
množica, imenujemo jo tudi \emph{univerzalna množica} algebre $A$, $\mathcal{F}$ pa tip algebre. Elemente $f \in 
\mathcal{F}$ imenujemo \emph{fundamentalne operacije} algebre $A$. 
\end{definicija}

Če je množica $\mathcal{F}$ končna, torej $\mathcal{F} = \left\{f_1, f_2, \dots, f_k\right\}$ običajno algebro $A$ 
zapišemo kot $\left(A, f_1, \dots, f_k\right)$. Fundamentalnim operacijam z rangom $1$ pravimo unitarne, za rangom $2$ 
pa binarne operacije.

Kot obljubljeno si sedaj definicijo oglejmo še na oprijemljivejši način z nekaj osnovnimi primeri, ki jih iz dosedanjega
študija že dobro poznamo.

\begin{primer}:
\begin{enumerate}
    \item Grupa $G$ je algebra $\left(G, \cdot, ^{-1}, 1\right)$ v kateri veljajo naslednje identitete:
    \begin{itemize}
        \item[G1] $x \cdot \left(y \cdot z\right) = \left(x \cdot y\right) \cdot z$
        \item[G2] $x \cdot 1 = 1 \cdot x = x$
        \item[G3] $x \cdot x^{-1} = x^{-1} \cdot x = 1$.
    \end{itemize} 
    Omenimo še, da je množenje: $\cdot$ binarna, invertiranje: $^{-1}$ pa unitarna operacija, enota $1$ je operacija 
    z rangom $0$. 
    \item Podobno so polgrupe algebre $\left(G, \cdot\right)$ in monoidi algebre  $\left(M, \cdot, 1\right)$.
    \item Tudi mreže, ki smo jih predstavili na začetku so algebre $\left(M,\vee, \wedge\right)$ v katerih veljajo
    identitete M1 - M4. 
    \item Kolobar je algebra $\left(R, +, \cdot, -, 0, 1\right)$, kjer sta $+$ in $\cdot$ binarni, $-$ unitarna in $0$ 
    ter $1$ z rangom 0. V kolobarju pa veljajo naslednji pogoji:
    \begin{enumerate}
        \item[R1] $\left(R, +, -, 0\right)$ je Abelova grupa,
        \item[R2] $\left(R, \cdot\right)$ je polgrupa, 
        \item[R3] $x \cdot \left(y + z\right) = \left(x \cdot y\right) + \left(x \cdot z\right)$ in
        $\left(x + y\right) \cdot z = \left(x \cdot z\right) + \left(y \cdot z\right)$
        \item[G2] $x \cdot 1 = 1 \cdot x = x$
    \end{enumerate}
\end{enumerate}
\end{primer}
Podobno lahko predstavimo še druge algebrske strukture npr. module in algebre nad kolobarjem.
Za formulacijo Mal'cevega izreka potrebujemo pojem kongruence, zato ga predstavimo.

\begin{definicija}
    Naj bo $A$ algebra tipa $\mathcal{F}$ in naj bo $\theta$ ekvivalenčna relacija na $A$. Tedaj je $\theta$ 
    \emph{kongruenca} na $A$, če zadošča naslednjemu pogoju: Za vsako $n$-arno operacijo $f \in \mathcal{F}$ in 
    vse elemente $a_i, b_i \in A$, če velja $a_i \theta b_i$ za vse $1 \leq i \leq n$, potem velja $f \left(a_1, \dots, 
    a_n\right) \theta f\left(b_i, \dots, b_n\right)$.
\end{definicija}

Ponovno si na primeru oglejmo kaj naj definicija pove, videli bomo, da kongruence v nekaterih primerih že dobro poznamo.

\begin{primer}
 Pokazali bomo, da so kongruence na grupah v bijektivni korespodenci s podgrupami edinkami. 
 Naj bo $G$ grupa in $\theta$ kongruenca na $G$. Označimo z $\left[1\right]_{\theta}$ ekvivalenčni razred enote v grupi 
 $G$ glede na ekvivalenčno relacijo $\theta$. Pokažimo, da je $\left[1\right]_{\theta}$ edinka grupe $G$. Vzemimo 
 poljubna elementa $a, b \in \left[1\right]_{\theta}$, torej zanju velja $a \theta 1$ in $b \theta 1$, produkt 
 $a b^{-1} = 1 1^{-1} = 1 \theta 1$ torej res leži v $\left[1\right]_{\theta}$. Zato je $\left[1\right]_{\theta}$ 
 podgrupa. Vzemimo sedaj poljuben element $g \in G$ in si oglejmo produkt $g h g^{-1}$ za neki $h \in 
 \left[1\right]_{\theta}$. Ker je $h$ ekvivalenten $1$ je $g h g^{-1} = g 1 g^{-1} = 1 \in \left[1\right]_{\theta}$. 
 Torej je $\left[1\right]_{\theta}$ res edinka v $G$.\\
 Obratno vzemimo $H$ podgrupo edinko grupe $G$. In definirajmo relacijo $\theta$ takole: $a \theta b \Leftrightarrow 
 a b^{-1} \in H$. Pokažimo, da je tako definirana relacija kongruenca.\\ 
 $\theta$ je res ekvivalnečna relacija: $(r)$ $aa^{-1}=
 1 \in H$, $(s)$ če je $ab^{-1} \in H$ za vsaka $a, b \in H$ je tudi $ba^{-1} \in H$ in še $(t)$ če sta $ab^{-1}$ in
 $bc^{-1}$ iz $H$ je tudi $ab^{-1} b c^{-1}= a c^{-1} \in H$. Pokažimo še, da je $\theta$ kongruenca. Vzemimo $a, a', 
 b, b'\in H$ take, da velja $a \theta a'$ in $b \theta b'$. Pokazati moramo, da velja $ab \theta a'b'$. Ker je $H$ 
 pogrupa obstajata taka $h_1, h_2 \in H$, da je $a' = ah_1$ in $b' = b h_2$, ker je $H$ edinka obstaja tak $h_3 \in H$, 
 da je $h_1b = bh_3$. Oglejmo si sedaj produkt $a'b' = ah_1bh_2 = abh_3h_2$, torej je $a'b' \theta ab$. 
 Torej je preslikava, ki kongruenci priredi podgrupo edinko: $\theta \mapsto \left[1\right]_{\theta}$ bijekcija.

\end{primer}

Za formulacijo Mal'cevega izreka moramo definirati še nekaj pojmov. 
\begin{definicija}
    \emph{Identiteta} tipa $\mathcal{F}$ nad $X$ je izraz oblike $p \approx q$, kjer sta $p, q \in T\left(X\right)$.
\end{definicija}
Pri tem je $T\left(X\right)$ množica vseh termov tipa $\mathcal{F}$ nad $X$, kjer je $X$ množica spremenljivk. Termi 
tipa $\mathcal{F}$ so izrazi, ki jih lahko sestavimo iz spremenljivk in operacij iz $\mathcal{F}$. Naprimer če za našo 
algebro vzamemo grupe so nekateri termi: $xy$, $x1$, $xyx^{-1}$, $x1xyxyy$. Podobno, če ima algebra tip $\mathcal{F} =
\left\{\cdot, +\right\}$, kjer sta $\cdot$ in $+$ binarni operaciji, so termi naprimer: $x+y, (x+y) \cdot z, xyz, \ldots$

Pravimo da algebra $A$ tipa $\mathcal{F}$ zadošča $p\left(x_1, \ldots, x_n\right)\approx q\left(x_1, \ldots, x_n\right)$,
če za vsako izbiro $a_1, \ldots, a_n \in A$ velja: $p\left(a_1, \ldots, a_n\right) \approx q\left(a_1, 
\ldots, a_n\right)$. To relacijo označimo takole: $A \models p \approx q$. Dodatno, pravimo da razred algeber $K$ 
zadošča $p \approx q$, če vsaka algebra iz $K$ zadošča $p \approx q$, kar označimo: $K \models p \approx q$.

Nadaljujmo z definicijama, ki bosta imeli pomembni vlogi pri našem izreku.

\begin{definicija}
    Algebra $A$ je \emph{kongruenčno-permutabilna}, če vsak par kongruenc komutira: $\theta_1 \circ \theta_2 =
    \theta_2 \circ \theta_1$, kjer sta $\theta_1$ in $\theta_2$ kongruenci na A. 
\end{definicija}

\begin{definicija}
    Neprazen razred $K$ algeber tipa $\mathcal{F}$ imenujemo \emph{raznoterost}, če je zaprt za podalgebre, 
    slike homomorfizmov in direktne produkte.
\end{definicija}

Sedaj smo pripravljeni, da brez dokaza navedemo ciljni izrek diplomske naloge.

\begin{izrek}[Mal'cev] Naj bo $V$ raznoterost tipa $\mathcal{F}$. Raznoterost $V$ je kongruenčno-permutabilna
    natanko takrat, ko obstaja term $p\left(x, y, z\right)$, da: \\ $$V \models p\left(x,x,y\right)\approx y$$ 
    in $$V \models p\left(x,y,y\right)\approx x.$$   
\end{izrek}

Za konec ponazorimo vsebino izreka z dvema preprostima primeroma.

\begin{primer}
    Grupe so kongruenčno-permutabilne, saj obstaja term $p(x,y,z)=xy^{-1}z$, ki zadošča navedenima identitetama:
    $p(x,x,y)= xx^{-1}y=y$ in $p(x, y, y)=xy^{-1}y=x$.\\
    Tudi kolobarji so kongruenčno-permutabilne algebre, za $p$ vzamemo $p(x,y,z)= x-y+z$.
\end{primer}

%\section{Literatura}

%\printbibliography
\end{document}