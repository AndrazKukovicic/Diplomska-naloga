\documentclass{beamer}
\usetheme{Madrid}
\usecolortheme{seahorse}
\setbeamertemplate{footline}{}
\usepackage{xcolor}
\definecolor{darkseagreen}{RGB}{143, 188, 143}
\setbeamercolor{item}{fg=gray}
\setbeamertemplate{footline}[frame number]
\newcommand{\FF}{$\mathcal{F}$ }
%Information to be included in the title page:
\title{Uvod v univerzalno algebro in Mal'cev pogoj}
\author{Andraž Kukovičič \\ {\small Mentorica: izr. prof. dr. Ganna Kudryavtseva \\}}

\institute{Fakulteta za matematiko in fiziko}
\date{2. 12. 2024}

\begin{document}

\frame{\titlepage}

\begin{frame}
\frametitle{Definicija algebre}

\emph{Tip} ali \emph{jezik} algebre $\mathbf{A}$ je množica funkcij oziroma operacij $\mathcal{F}$,  $f \in \mathcal{F}, f:\mathbf{A}^n \rightarrow \mathbf{A}$.
Vsaka funcija $f \in 
\mathcal{F}$ ima prirejeno nenegativno celo število $n$, ki ga imenujemo \emph{arnost} ali 
\emph{rang} $f$. \\
\pause
\begin{block}{}

\emph{Algebra} $\mathbf{A}$ tipa $\mathcal{F}$ je urejeni par $ \left(A, F\right)$, kjer je $A$ neprazna množica 
in $F$ družina operacij s končnim številom argumentov na $A$, indeksirana s tipom \FF tako, 
da je vsakemu $n$-arnemu funkcijskemu simbolu $f$ iz \FF prirejena $n$-arna operacija $f^\mathbf{A}$ 
na $A$, ki jo imenujemo fundamentalna operacija algebre $\mathbf{A}$. \\

\end{block}
\pause
Množico $A$ imenujemo univerzalna množica algebre $\mathbf{A}  = \left(A, F\right)$. \\
\end{frame}

\begin{frame}
    \frametitle{Primeri algeber}
    \begin{itemize}
        \item \textcolor{darkseagreen}{Grupa} $\mathbf{G}$ je algebra $\left(G, \cdot, ^{-1}, 1\right)$ v kateri veljajo naslednje
        \item[] identitete:
        \begin{itemize}
            \item[G1] $x \cdot \left(y \cdot z\right) = \left(x \cdot y\right) \cdot z$
            \item[G2] $x \cdot 1 = 1 \cdot x = x$
            \item[G3] $x \cdot x^{-1} = x^{-1} \cdot x = 1$.
        \end{itemize}
    \pause
        \item \textcolor{darkseagreen}{Polgrupe} so algebre $\left(G, \cdot\right)$
        \item \textcolor{darkseagreen}{Monoidi} so algebre  $\left(M, \cdot, 1\right)$
        \item \textcolor{darkseagreen}{Mreže} so algebre $\left(L,\vee, \wedge\right)$ 
    \end{itemize}
\end{frame}

\begin{frame}
\frametitle{Mreže}
    
    Neprazna množica $M$ z dvema binarnima operacijama, ki ju označimo z $\vee$ in $\wedge$ je mreža, če zadošča:\\
    \begin{enumerate} 
        \item[M1] $x \vee y = y \vee x$
        \item[M2] $x\vee \left(y \vee z\right) = \left(x \vee y\right)\vee z$
        \item[M3] $x \vee x = x$
        \item[M4] $x = x \vee \left(x \wedge y\right)$. \\
    \end{enumerate}
    \pause
    \textcolor{darkseagreen}{Ekvivalentno:} \\
        Delno urejena množica $M$ je mreža natanko takrat, ko za vsaka $a, b \in M$ obstajata $sup\{a, b\}$ 
        in $inf\{a, b\}$.\\
        \textcolor{darkseagreen}{Primer:}\\
    $M = \mathbb{N}$, z $\vee$ označimo najmanjši skupni večkratnik, z $\wedge$ pa največji skupni delitelj
    števil. Potem je $\left(\mathbb{N}, \vee, \wedge\right)$ mreža. 
    
\end{frame}
\begin{frame}
\frametitle{Kongruence}
\begin{block}{}
Naj bo $\mathbf{A}$ algebra tipa $\mathcal{F}$ in naj bo $\theta$ ekvivalenčna relacija na $A$. Tedaj je $\theta$ \emph{kongruenca} 
na $\mathbf{A}$, če zadošča naslednjemu pogoju: \\
za vsak $n$-arni funkcijski simbol $f \in \mathcal{F}$ in elemente $a_i, b_i \in A$, če velja $a_i\theta b_i$ 
za vse $1 \leq i \leq n$, potem velja $f^\mathbf{A}\left(a_1, \dots, a_n\right) \theta 
f^\mathbf{A}\left(b_1, \dots, b_n\right)$.\\
\end{block}

Z $Con \mathbf{A}$ označimo množico vseh kongruenc algebre $\mathbf{A}$.
\\
\pause
Če $Con \mathbf{A}$ opremimo z operacijama $\wedge$ in $\vee$, ki sta definirani:\\
$\theta_1 \wedge \theta_2 = \theta_1 \cap \theta_2$ in \\ 
$\theta_1 \vee \theta_2 = \theta_1 \cup \left(\theta_1 \circ \theta_2\right)
\cup \left(\theta_1 \circ \theta_2 \circ \theta_1\right) \cup \left(\theta_1 \circ \theta_2 \circ \theta_1 \circ \theta_2\right) \cup \dots$ \\
postane $Con \mathbf{A}$ mreža.\\
Pri tem je $\theta_1 \vee \theta_2$ najmanjša kongruenca, ki vsebuje $\theta_1$ in $\theta_2$.
\end{frame}

\begin{frame}
    \textcolor{darkseagreen}{Primer:}\\
Naj bo $\mathbf{G}$ grupa. \\
Če je $\theta \in Con \mathbf{G}$ potem z $\left[1\right]_\theta$ označimo ekvivalenčni razred enote $1$.\\
$\left[1\right]_\theta$ je univerzalna množica podgrupe edinke grupe $\mathbf{G}$. \\~\\ 
Za poljubna $a, b \in G$ velja $\left(a, b\right) \in \theta \Leftrightarrow a \cdot b^{-1} \in \left[1\right]_\theta$.\\
Če je $\mathbf{N} \vartriangleleft \mathbf{G}$, potem je relacija na $G$ definirana z:
$\left(a,b\right) \in \theta \Leftrightarrow a \cdot b^{-1} \in N$ kongruenca na $\mathbf{G}$ in $\left[1\right]_\theta= N$.\\~\\
Preslikava s predpisom $\theta \mapsto \left[1\right]_\theta$ je bijekcija med kongruencami na $\mathbf{G}$ in edinkami grupe $\mathbf{G}$.\\
\end{frame}
\begin{frame}
\frametitle{Identiteta}
    \begin{block}{}
        \emph{Identiteta} tipa $\mathcal{F}$ nad $X$ je izraz oblike $p\approx q$, kjer sta $p, q \in T\left(X\right)$. 
    \end{block}
    $T\left(X\right)$ je množica vseh termov tipa $\mathcal{F}$ nad $X$. Kjer je $X$ množica spremenljivk.\\~\\
    Algebra $\mathbf{A}$ tipa $\mathcal{F}$ zadošča $p\left(x_1, \ldots, x_n\right)\approx q\left(x_1, \ldots, x_n\right)$,
    če za vsako izbiro $a_1, \ldots, a_n \in A$ velja: $p^\mathbf{A}\left(a_1, \ldots, a_n\right) \approx q^\mathbf{A}\left(a_1, \ldots, a_n\right)$.\\
    Oznaka: $\mathbf{A} \models p\approx q$.\\~\\
    Razred algeber $K$ zadošča $p\approx q$, če vsaka članica $K$ zadošča $p\approx q$. \\
    Oznaka: $K \models p\approx q$.
\end{frame}
\begin{frame}
\frametitle{Mal'cev izrek}
Algebra $\mathbf{A}$ je \emph{kongruenčno-permutabilna}, če vsak par kongruenc komutira: \\
$\theta_1 \circ \theta_2 = \theta_2 \circ \theta_1$.\\
Neprazen razred $K$ algeber tipa $\mathcal{F}$ imenujemo \emph{raznoterost}, če je zaprt za podalgebre, 
slike homomorfizmov in direktne produkte.
\pause
\begin{block}{}
\textcolor{darkseagreen}{Izrek:} Naj bo $V$ raznoterost tipa $\mathcalc{F}$. Raznoterost $V$ je kongruenčno-permutabilna
natanko takrat, ko obstaja term $p\left(x, y, z\right)$, da: \\ $$V \models p\left(x,x,y\right)\approx y$$ in $$V \models p\left(x,y,y\right)\approx x.$$
\end{block}\\
\textcolor{darkseagreen}{Primera:} Grupe, kolobarji.


\end{frame}

\end{document}
